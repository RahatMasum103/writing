% !TeX program = latexmk
%%%
%%% LaTeX 2e (mostly) version of TTU thesis and dissertation format
%%% by Mike Renfro (renfro at tntech.edu) et al
%%%
%%% Basic instructions for document and formatting options are
%%% included inline throughout the .tex files -- read them thoroughly,
%%% and avoid changing anything other than what is specifically listed
%%% as editable.

%%% On the \documentclass line below, one thing you might need to
%%% change would be the font size. 10pt, 11pt, and 12pt are all
%%% accepted by the Graduate School, but your advisor and committee
%%% may be more restrictive. The default of 12pt should be fine unless
%%% you have a very large committee and a long thesis or dissertation title,
%%% which would overflow the approval sheet. You may also want to add the
%%% draft option, which will reformat your thesis to take up fewer sheets
%%% during the early editing process. Finally, you may want to add the
%%% copyrighted option, which will add a copyright page to the front matter.
\documentclass[11pt,twoside]{ttuthesis}

\usepackage{epstopdf}
\usepackage{subcaption}
\usepackage{adjustbox}
\let\newfloat\undefined
%\usepackage{floatrow}

\usepackage[tableposition=top]{caption}
%subfigure,,endnotes,footnote,,multicol,sidecap,url,colortbl,booktabs,,epsfigalgorithm,algorithmic,,amsmathmultirow,enumerate
%\usepackage{algcompatible,algorithm,algpseudocode,amssymb,algorithmicx}
%\usepackage{algorithm,algorithmicx}
\usepackage{amsfonts}
%\usepackage [english]{babel}
%\usepackage{balance}
\usepackage{amsthm}
\usepackage{algorithm,algpseudocode,amssymb}
\usepackage{mathtools}
\usepackage{environ}
\theoremstyle{plain}
%\pagenumbering{roman}
\newtheorem{theorem}{Theorem}
\newtheorem{definition}{Definition}
%\renewenvironment{proof}{\noindent{\bfseries Proof}}{\qed}
%\let\oldproofname=\proofname
%\renewcommand{\proofname}{\rm\bf{\oldproofname}}
%\newproof{proof}{Proof}
%\usepackage{color}
%\definecolor{red}{rgb}{1,0,0}
%\newcommand{\remark}[1]{$\rightarrow$ \textbf{\textit{\color{red} Remark: }} \textit{\color{blue} #1} $\leftarrow$}
%\addto\captionsenglish{\renewcommand{\figurename}{Fig.}}
\newcommand{\eg}{\textit{e.g.}}
\newcommand{\ie}{\textit{i.e.}}
\newcommand{\etal}{\textit{et al.}}
%\newcommand{\ts}{\textsuperscript}
%%% For all notes below, let JOBNAME equal the base filename of this
%%% main .tex file for your thesis/dissertation (that is, this file you're
%%% reading right now). The macro \jobname has already been set to this value,
%%% and is used in several places below. There is no need to change this unless
%%% you decide to totally rearrange how various content and settings are stored.
%%%
%%% Go ahead and save this file under a new name (thesis-your-last-name.tex or
%%% thesis-your-ttu-username.tex, for example). That way, if you happen to
%%% update to a new version of the thesis files, your changes won't be
%%% overwritten. Also, make a new subdirectory in this directory named
%%% similarly (for example, thesis-your-last-name-content or
%%% thesis-your-ttu-username-content) to hold your chapters, figures, and other
%%% related items.

%%% Notes on adding other packages to your thesis:
%
% 1. The memoir class that ttuthesis.cls is based from already
%    emulates many packages' features. There is no need to load any of the
%    following packages at all---simply use their commands as if you'd already
%    loaded them:
%
%    appendix, array, booktabs, ccaption, chngcntr, crop, dcolumn, delarray,
%    enumerate, epigraph, framed, ifmtarg, ifpdf, index, makeidx, moreverb,
%    needspace, newfile, nextpage, pagenote, patchcmd, parskip, setspace,
%    shortvrb, showidx, tabularx, titleref, tocbibind, tocloft, verbatim,
%    verse.
%
% 2. The memoir class also provides functions equivalent to those in the
%    following packages, although it does not prevent you from loading them
%    via \usepackage:
%
%    fancyhdr, geometry, sidecap, subfigure, titlesec.
%
% 3. ttuthesis.cls automatically loads the following packages. There is
%    no need to re-load them in your own files:
%
%    hypcap, hyperref, ifthen, indentfirst, listings, memhfixc, nomencl,
%    refcount, rotating, ted.
%
% 4. For any remaining packages you need to load, some may need to be
%    loaded before hyperref, and some may need to be loaded after hyperref.
%    Add your entries to JOBNAME-packages-loaded-before-hyperref.sty and
%    JOBNAME-packages-loaded-after-hyperref.sty as appropriate.

%%% Your thesis or dissertation title. Insert line breaks manually
%%% with the \\ characters so that when printed, the title has an
%%% inverted pyramid format. If your title contains Greek letters,
%%% superscripts, subscripts, or other typography that can't be
%%% rendered as a PDF string, use the command
%%% \title{\texorpdfstring{TeX title}{PDF title}} instead.
\title{Continuous Surveillance Design for Critical Smart Grid Infrastructure using Unmanned Aerial Vehicle (UAV)}

%%% Your name, as registered at the university.
\author{Rahat Masum}

%%% This example and style file was designed to render into a PDF file
%%% rather than DVI or Postscript, and for all cross-references (to
%%% specific chapters, pages, equations, references, etc.) to be
%%% hyperlinks in that PDF. The hyperref package handles the
%%% hyperlinks, and also the basic metadata for the PDF, such as the
%%% author, keywords, etc.
%%%
%%% ttuthesis.cls automatically fills out the fields for pdfauthor and
%%% pdftitle. The pdfkeywords and pdfsubject fields are optional, but
%%% may be useful in the future if we do an electronic archive of
%%% theses and dissertations and want to make it easier to search. If
%%% you don't fill out accurate subject and keywords fields, please
%%% remove or comment the following hypersetup command entirely.
%%%
\hypersetup{
  pdfsubject={Format and style rules for theses and dissertations at Tennessee Technological University},
  pdfkeywords={thesis, dissertation, style guide}
}

%%%
%%% Other information relevant to the front matter of your thesis.
%%%

%%% Your abstract.
\abstract{%

A smart grid is a widely distributed engineering system with overhead transmission lines often running through deep forests, long rivers, coastal and hilly areas, and busy cities. Physical damage to those power lines, from natural calamities or technical failures, will disrupt the functional integrity of the grid. To ensure system recovery and the continuation of secure operational flow when those phenomena happen, the grid operator must immediately take steps to nullify the impacts and repair the problems, even if those occurs in hardly reachable remote areas. Emerging unmanned aerial vehicles (UAVs) show great potential to replace traditional human patrols for regularly monitoring critical situations involving the safety of the grid. The critical lines can be monitored by a fleet of UAVs to ensure a resilient surveillance system. The proposed approach first considers the \textit{n}-1 contingency analysis to find the criticality of a transmission line from its performance index, which is calculated using linear sensitivity factors. We divide a line into smaller segments forming multiple inspection points on the line. Then, we propose a formal framework that verifies whether a given set of UAVs can perform continuous surveillance of the grid satisfying various requirements, particularly the monitoring and resiliency specifications. The verification process ultimately provides a trajectory plan for the UAVs, including the refueling schedules. The resiliency requirement of inspecting a point is expressed in terms of a \textit{k}-property specifying that if \textit{k} UAVs fail or compromised still there is a UAV to collect the data at the point within a threshold time, while the points that are under resilient surveillance cover a required percentage of the grid’s overall criticality. We evaluate the proposed framework on synthetic data based on various IEEE test bus systems. With the solution model, we have implemented a graphical simulation UX using Unity3D to present the routing of UAVs as a surveillance scenario.
}

%%% Is this document and thesis or a dissertation?
\doctype{Thesis}

%%% Your planned degree
\degree{Master of Science}

%%% Your major. Engineering PhD students would just put Engineering
%%% here, not a specific department's name.
\department{Computer Science}

%%% The month and year of your graduation.
\graduationmonth{May}
\graduationyear{2019}

%%% The text of your dedication page, if you have one.
\dedication{%
  \begin{center}
    This thesis is dedicated to my parents,\\
who have always been to my support during the dark nights, the sleepless mornings and the frustrated hours of insane pressurized situations. Without their kind words and unconditional love, I would not be able to cross the journey.
  \end{center}
}

%%% The text of your acknowledgments page.
\acknowledgments{%
  I am grateful to my supervisor Dr. Mohammad Ashiqur Rahman for guiding me with proper directions to this work. It was my pleasure to work with such a hard-working person who had got the best out of me, at the same time allowed me to make this paper to be my own work. I am thankful to Dr. Ambareen Siraj for all the support she has been providing me from the very beginning of my Masters program. My heartiest respect to Dr. Mike Rogers for giving his kind consent and time to be the chairperson of my committee. I want to mention my student colleague Mr. Jakaria, Mr. Mehedi, and Mr. Amarjit for always cheering me with their friendly behavior and helping attitude whenever I became stuck at any problem. I am grateful to my friend Mr. Hasan Shahriar to help me with Matlab analysis. I would like to appreciate the overall academic and administrative support I have got from Dr. Ghafoor and our honorable Chair of the department Dr. Gannod. I express my gratitude to the Administrative Associate of our department Ms. Megan Cooper for her continuous cooperation from the first day of my journey at Tennessee Tech.
}
%%%
%%% Information about your committee.
%%%
%%% Note: you may have to switch to a smaller font if your
%%% thesis/dissertation title is very long and your committee is very large.
%%% Using a 10pt font, a title seven lines long only allows for six people
%%% on the committee (including chairs and cochairs). A title two lines long
%%% allows for up to nine people on the committee (including chairs and
%%% cochairs).

%%% First, who is your committee chair?
\committeechair{Michael Rogers}

%%% Second, does your committee have a single chair, or two cochairs?
%%% If you have two cochairs, use one name in \committeechair, and the
%%% second name in \committeecochair. If you have no cochair, leave the
%%% \committeecochair command commented out.
%\committeecochair{Cochair's Name}

%%% Third, what are the names of your committee members?
\committeemembers{Mohammad Ashiqur Rahman, Ambareen Siraj}

\begin{document}

%%%
%%% Typesetting your document's front matter. Most of this is automated.
%%%
%%% Abstract page, title page: mandatory, automatically added to all
%%% theses and dissertations. Use the \abstract, \title, \author, \doctype,
%%% \degree, \department, \graduationmonth, and \graduationyear commands
%%% above to edit the page content. The page format and location is fixed,
%%% and cannot be edited.
%%%
%%% Copyright page: optional, automatically added to all theses and
%%% dissertations with copyrighted in the \documentclass options above. The
%%% page format and location is fixed, and cannot be edited.
%%%
%%% Approval page: mandatory, automatically added to all theses and
%%% dissertations. Placed immediately after the copyright page (if present)
%%% or the title page (if no copyright page is present).  The page format
%%% and location is fixed, and cannot be edited.
%%%
%%% Dedication page: optional, automatically added to all theses and
%%% dissertations with a \dedication command. Placed immediately after the
%%% approval page. The \dedication command defines the content of the 
%%% dedication page. The page format and location is fixed, and cannot be
%%% edited.
%%%
%%% Acknowledgments page: optional, automatically added to all theses and
%%% dissertations with a \acknowledgments command. Placed immediately after the
%%% dedication page (if present) or the approval page (if no dedication page
%%% is present). The \acknowledgments command defines the content of the 
%%% acknowledgments page. The page format and location is fixed, and cannot be
%%% edited.

%%% The table of contents and lists of figures, tables, etc. Except
%%% for commenting out the lists of tables, figures, or other elements
%%% missing from your document, you shouldn't need to change any of
%%% the table of contents commands.
\begin{Spacing}{1}
\tableofcontents*  % Leave the * after this command to keep the table of
                   % contents from appearing in the table of contents
\listoftables      % Tables
\listoffigures     % Figures
\lstlistoflistings % Program Listings, from the listings package
%%% If you want to rename your list of symbols, edit the
%%% \renewcommand{\nomname} command accordingly.
\renewcommand{\nomname}{LIST OF SYMBOLS}
\printnomenclature % Symbols
\end{Spacing}

%%% Leave the \mainmatter and command alone. This resets the page numbering
%%% and styles from what's appropriate for the front matter to what's
%%% appropriate for the body of your thesis/dissertation.
\mainmatter

%%% Edit the \include commands below to match the base filenames of
%%% your chapter files. Do *not* add a .tex extension to the filename
%%% -- chapter1.tex is included with a \include{chapter1} command.
\include{Rahat-thesis-content/introduction}
\include{Rahat-thesis-content/background}
\include{Rahat-thesis-content/solution_method}
\include{Rahat-thesis-content/line_analysis}
\include{Rahat-thesis-content/formal_model}
\include{Rahat-thesis-content/implementation}
\include{Rahat-thesis-content/evaluation}
\include{Rahat-thesis-content/simulation}
%\include{Rahat-thesis-content/future_works}
\include{Rahat-thesis-content/conclusion}
%\include{\jobname-content/experimental}
%\include{\jobname-content/results}
%\include{\jobname-content/conclusions}

%%% Bibliography settings
\renewcommand{\bibname}{References}
\include{Rahat-thesis-content/bibliography}

%%%
%%% Appendices settings
%%%

%%% Does your thesis have one appendix, or multiple appendices?
%%% Uncomment or comment each of the next two lines accordingly.
%\renewcommand{\appendixtocname}{Appendix} % I have only one appendix
%\renewcommand{\appendixtocname}{Appendices} % I have multiple appendices

%%% If you have any appendices on your thesis, leave the \appendix and
%%% command alone. It resets the ToC style to label your appendices by
%%% letter rather than number, and typesets the required separation page
%%% between the Appendices and the main body text. If you have no
%%% appendices, comment the \appendix command.
\appendix

%%% Edit the \include commands below to match the relative path and base
%%% filenames of your appendix files. Do *not* add a .tex extension to the
%%% filename -- appendix-a.tex is included with a \include{appendix-a}
%%% command. Obviously, if your thesis has no appendices, then comment them
%%% all.

\include{Rahat-thesis-content/code-listings}
\include{Rahat-thesis-content/algorithms}

%%% Edit the \vita command below to match the relative path and base
%%% filename of your vita. Do *not* add a .tex extension -- vita.tex is
%%% included with a \vita{vita} command.
\vita{Rahat-thesis-content/vita}

\end{document}
